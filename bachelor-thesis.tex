\documentclass[specification,annotation]{itmo-student-thesis}

%% Опции пакета:
%% - specification - если есть, генерируется задание, иначе не генерируется
%% - annotation - если есть, генерируется аннотация, иначе не генерируется
%% - times - делает все шрифтом Times New Roman, собирается с помощью xelatex
%% - languages={...} - устанавливает перечень используемых языков. По умолчанию это {english,russian}.
%%                     Последний из языков определяет текст основного документа.

%% Делает запятую в формулах более интеллектуальной, например:
%% $1,5x$ будет читаться как полтора икса, а не один запятая пять иксов.
%% Однако если написать $1, 5x$, то все будет как прежде.
\usepackage{icomma}

%% Один из пакетов, позволяющий делать таблицы на всю ширину текста.
\usepackage{tabularx}

\newcolumntype{L}[1]{>{\hsize=#1\hsize\raggedright\arraybackslash}X}%
\newcolumntype{R}[1]{>{\hsize=#1\hsize\raggedleft\arraybackslash}X}%
\newcolumntype{C}[1]{>{\hsize=#1\hsize\centering\arraybackslash}X}%

%% Данные пакеты необязательны к использованию в бакалаврских/магистерских
%% Они нужны для иллюстративных целей
%% Начало
\usepackage{tikz}
\usetikzlibrary{arrows}
%% Конец

%% Указываем файл с библиографией.
\addbibresource{bachelor-thesis.bib}

\graphicspath{{./media/}}

\begin{document}

\studygroup{K3400}
\faculty{ИКТ}
\specialty{11.03.02~Инфокоммуникационные технологии и системы свзяи}
\specialization{Программно-защищённые инфокоммуникации}
\title{Разработка программы автоматизированного построения полётного задания для
управления группой БПЛА}
\author{Де Ла Пенья Смирнов Ярослав}{Де Ла Пенья Смирнов Я.}
\supervisor{Карманов Андрей Геннадиевич}{Карманов А.Г.}{доцент, к.т.н.}{}
\programhead{Присяжнюк С.П.}{профессор, д.т.н.}
\publishyear{2020}
%% Дата выдачи задания. Можно не указывать, тогда надо будет заполнить от руки.
\startdate{20}{декабря}{2019}
%% Срок сдачи студентом работы. Можно не указывать, тогда надо будет заполнить от руки.
\finishdate{28}{мая}{2020}
%% Дата защиты. Можно не указывать, тогда надо будет заполнить от руки.
\defencedate{19}{июня}{2020}

%% \addconsultant{}{звание}

%% \secretary{}

%% Задание
%%% Техническое задание и исходные данные к работе
\technicalspec{Требуется спроектировать и разработать прототип программного
обеспечения для автоматизированного построения полётного задания для управления
группой беспилотных летательных аппаратов.}

%%% Содержание выпускной квалификационной работы (перечень подлежащих разработке вопросов)
\plannedcontents{Изучение теории проектирования и разработки систем беспилотных
летательных аппаратов; изучение и анализ необходимых компонентов для
проектирования и разработки программного обеспечения автоматизированного
построения полётного задания для управления группой беспилотных летательных
аппаратов; проектирование и разработка программного обеспечения; описание
архитектуры программного обеспечения.}

\plannedgraphics{Диаграммы, таблицы, рисунки в описательной части; схемы
алгоритмов и листинги кода.}

%%% Исходные материалы и пособия
\plannedsources{\begin{enumerate}
    \item Reg Austin. Unmanned aircraft systems : UAVs design, development and
      deployment;
    \item Douglas M. Marshall и др. Introduction to Unmanned Aircraft Systems;
    \item ресурсы сети интернет: PX4, Gazebo, Cesium, ECMA International.
\end{enumerate}}

%% Аннотация
%%% Цель исследования
\researchaim{Разработать прототип программного обеспечения (ПО) для
автоматизированного построения полётного задания группы БПЛА.}

%%% Задачи, решаемые в ВКР
\researchtargets{\begin{enumerate}
  \item исследовать и изучить предметную область и теоретические основы темы;
  \item определить общие требования к ПО;
  \item выбрать и изучить оптимальные компоненты и библиотеки для реализации ПО;
  \item выбрать оптимальный язык программирования для реализации ПО;
  \item спроектировать и разработать архитектуру для взаймодействия разных
    компонентов;
  \item разработать прототип ПО для построения полётного задания для управления
    группой БПЛА;
\end{enumerate}}

%%% Использование современных пакетов компьютерных программ и технологий
\addadvancedsoftware{Симулятор робототехники Gazebo}{Библиографический список}
\addadvancedsoftware{Автопилот PX4}{Библиографический список}

%%% Краткая характеристика полученных результатов
\researchsummary{<++>}

%%% Гранты, полученные при выполнении работы
\researchfunding{нет}

%%% Наличие публикаций и выступлений на конференциях по теме выпускной работы
\researchpublications{нет}

%% Эта команда генерирует титульный лист и аннотацию.
\maketitle{Бакалавр}

%% Оглавление
\tableofcontents

%% Макрос для введения. Совместим со старым стилевиком.
\startprefacepage

Последние годы уровень развития технологий, в том числе информационных
технологий позволило расширить спектр задач в которых применяются беспилотные
летательные аппараты (БПЛА или БЛА). Изначально БПЛА развивались почти
исключительно для военных целей, только с недавних пор они начали
разрабатываться для гражданских, коммерческих и других государственных целей.

БПЛА --- летательный аппарат (ЛА), который выполняет полёт без человека на борту.
Полётом можно управлять удалённо с наземной станции, либо же в полностью
автоматическом режиме по заранее заданным траекториям или другим параметрам.

БПЛА обычно обладают теми же самыми элементами как и обычные ЛА, отличаясь лишь
в отсутствии экипажа на борту летательного аппарата. Другие элементы, то есть ---
запуск, посадка, возвращение, коммуникации, поддержка, и т.д. имеют свои аналоги
и в пилотируемых и в беспилотных ЛА.\cite{austin-uas}

Разные виды БПЛА имеют разную степень автономности в зависимости от способностей
ЛА и от требования к задачи. Среди разных видов управления БПЛА в зависимости от
автономности или автоматизированности есть\cite{douglas-intro-to-uas}:

\begin{itemize}
  \item полностью ручное --- когда оператор управляет БПЛА напрямую манипулируя
    движением и траекторией ЛА;
  \item стабилизированное --- когда оператор подаёт команды для управления БПЛА и
    эти в свою очередь обрабатываются автопилотной или компьютерной системой на
    борту и преобразуются в нужные действия;
  \item автоматизированное --- в автоматизированном способе управления оператор
    обладает косвенным контролем над БПЛА, задавая нужные параметры полёта во
    время полёта или заранее составляя полётное задание. Обычно такой вид
    управления производиться посредством ПО с графическим или иным интерфейсом.
\end{itemize}

Нельзя сразу сказать что у БПЛА всегда есть преимущество или недостаток по
сравнению с пилотируемыми летательными аппаратами (ПЛА). Всё зависит от задачи.
Традиционно использование БПЛА, и других роботизированных машин, относили к
задачам или работам, которые в англоязычном мире принято обозначать DDD -- dull,
dirty and dangerous, что на русский язык переводится как «муторно, грязно и
опасно». Данные критерии до сих пор актуальные, но они больше не являются
единственными, которые решают преимущество БПЛА для конкретной задачи. Следующие
виды задач можно отнести к задачам где у БПЛА существует преимущество:

\begin{itemize}
  \item муторные или скучные задачи --- как в военных как и в гражданских
    отраслях существуют задачи как например, длительное наблюдение, которые
    после определенного времени могут привести к усталости и соответственно к
    снижению сосредоточенности и продуктивности что в последствии приводит к
    потери эффективности задачи.
  \item грязные задачи --- также относящийся к военных и гражданских целей, такие
    задачи как исследование среды для выявления загрязнения, например
    радиоактивной или химической, избегая подвержения риску человека. К тому же,
    деконтаминация летательного аппарата проще в случае БПЛА.
  \item опасные задачи --- если говорить о военных задачах, то один из примеров
    это конечно разведка в территориях оккупированными врагами. В таких случаях
    не только избегания риска людей дают преимущество, но и малогабаритность
    БПЛА усложняют их обнаружение или поражение. В случае гражданских задач, то
    это например исследование линии электропередачи, или в борьбе с лесными
    пожарами. Также можно отметить ситуации с экстремальными погодными
    условиями.
  \item скрытые или тайные задачи --- в связи с тем что БПЛА часто меньше по
    размеру чем ПЛА, то их соответственно сложнее обнаружить. Такая
    характеристика БПЛА полезная когда нужно следить за врагом, таким образом
    чтобы он не знал об этом.
  \item исследовательские задачи --- самые первые БПЛА как раз и использовались
    для исследования и разработки новых авиационных систем. Очень часто
    используются копии новых разработанных прототипов гражданских и военных ЛА,
    в том числе ПЛА, что позволяет их тестировать в реалистичных условиях
    дешевле и с меньшей опасностью.
\end{itemize}

Стоит ещё отметить один большой аргумент в пользу БПЛА, и это экономические
причины. Обычно БПЛА, в тех же самых задачах или ролях, меньше ПЛА по размеру,
что и сказывается на их более низкую первоначальную стоимость. Затраты на
эксплуатацию БПЛА тоже ниже, так как их обслуживать проще и дешевле, они тратят
меньше топлива, и стоимость их хранение тоже меньше. Быстрое развитие
информационных технологии также способствовало к значительному снижению
стоимости и большей доступности тех компонентов, которые делают возможным
беспилотный режим полёта.

\textbf{Актуальность темы} данной выпускной квалификационной работы
заключается в том что в современном мире всё больше и больше растёт спрос на
беспилотных ЛА и появляются новые применения для них. С ростом спроса на БПЛА
появляются новые задачи где требования к возможностям беспилотных систем
меняются и усложняются. Уже не является достаточным запуск одного аппарта и его
управления с наземной станции. Новые задачи требуют больше автономности, не
только в рамках заданного полётного задания, но в том числе способность БПЛА
принимать решения и координировать свои действия с другими БПЛА.

Новые разработки в сфере информационных технологии открывают новые возможности
для дальнейшего развития беспилотных систем для удовлетворения современным
требованиям. В том числе открывается возможность для автоматизированного
построения полётного задания для групп БПЛА.

\textbf{Цель работы} состоит в том чтобы разработать прототип программного
обеспечения (ПО) для автоматизированного построения полётного задания группы
БПЛА. Для выполнения данной цели необходимо решить следующие задачи:

\begin{itemize}
  \item исследовать и изучить предметную область и теоретические основы темы;
  \item определить общие требования к ПО;
  \item выбрать и изучить оптимальные компоненты и библиотеки для реализации ПО;
  \item выбрать оптимальный язык программирования для реализации ПО;
  \item спроектировать и разработать архитектуру для взаймодействия разных
    компонентов;
  \item разработать прототип ПО для построения полётного задания для управления
    группой БПЛА;
\end{itemize}

%% Начало содержательной части.
\chapter{Обзор предметной области и анализ задания}\label{ch:overview}

\section{Обзор систем беспилотных летательных аппаратов}\label{sec:sys-overview}

\subsection{Элементы системы беспилотного летательного
аппарата}\label{subsec:uas-elements}

Начать обзор стоит с системы беспилотного летательного аппарата и каждого
элемента данной системы. Большинство гражданских беспилотных систем состоят из
удалённо пилотируемого ЛА, человеческого фактора, полезного груза, подсистемы
управления, и архитектуры и канала передачи данных и коммуникаций.
~\cite{douglas-intro-to-uas} В военных системах конечно могут присутствовать и
другие элементы, как например оружейные системы.  Рисунок~\ref{pic:diag1} более
наглядно показывает простой пример системы БПЛА.

\begin{figure}[!h]
  \caption{Элементы системы беспилотного летательного аппарата}\label{pic:diag1}
  \centering
  \includegraphics[width=0.9\textwidth]{diag1}
\end{figure}

Далее описаны и определены основные элементы системы БПЛА:

\begin{itemize}
  \item \textbf{БПЛА} --- под беспилотным летательным аппаратом подразумевается
    летательный аппарат, управляемый без прямого вмешательства человека внутри
    или на борту данного аппарата. Термин «беспилотный» на самом деле не совсем
    хорошо описывает данные аппараты, поскольку в большинство случаев человек
    всё-таки держит контроль над ЛА, и уровень вмешательства человека в
    управлении БПЛА сильно варьирует от одной системы к другой.
  \item \textbf{Подсистемы управления} --- систему управления можно поделить на
    две части, на бортовую систему управления (автопилот), и на наземную
    станцию управления (НСУ).
    \begin{itemize}
      \item \textbf{Автопилот} --- одно из элементарных возможностей БПЛА это
        выполнение задания минимизируя участие оператора в процессе. Система
        автопилота обеспечивает данную возможность. Степень автономности сильно
        варьируется в зависимости от модели ЛА. Начиная с самых примитивных
        систем где ЛА напрямую управляется оператором и автопилот лишь помогает
        стабилизировать полётные действия, заканчивая с системами где автопилот
        полностью управляет полётом со взлёта до посадки без участия оператора;
        оператор может взять под контроль БПЛА в экстренном случае. В последнее
        время стали массово доступные коммерческие системы автопилота в том
        числе и для малых БПЛА. Многие из этих систем также являются открытыми
        (open source), или предоставляют возможность замена компонентов на
        открытые, позволяя таким образом настроить систему полностью под свои
        нужды.
      \item \textbf{НСУ} --- наземные станции управления предоставляют средства
        для управления БПЛА человеком. Они бывают разных размеров, от ручных
        передатчиков (рисунок~\ref{pic:rc-tx}), до больших объектов которые состоят из
        несколько сооружений с разными приборами и рабочими станциями
        (рисунок~\ref{pic:gcs}).
    \end{itemize}
  \item \textbf{Канал передачи данных} --- этот термин применяется для
    обозначения способа связи между НСУ и автопилотом и описывает как информация
    для управления БПЛА отправляется и получается между ними. Иногда могут
    применяется отдельные каналы связи для коммуникации с системами полезных
    грузов. В отношении радиосвязи, каналы передачи данных можно разделить на
    два вида --- прямой видимости и вне прямой видимости.
    \begin{itemize}
      \item \textbf{Прямая видимость} --- операции или связи прямой видимости
        это прямая связь между НСУ и бортом используя радиоволны. Обычно в
        гражданских БПЛА при радиосвязи прямой видимости используются
        радиочастотные диапазоны 433 МГц, 2,45 ГГц и 5,8
        ГГц.~\cite{avaks-comm-systems} Используемый частотный диапазон будет
        зависит от параметров полёта, используемое оборудование и требований
        полётного задания.
      \item \textbf{Вне прямой видимости} --- связи вне прямой видимости
        производятся обычно с помощью спутниковых коммуникации, или используя
        релейные станции и/или транспорт. Большинство малых гражданских БПЛА не
        нуждаются или не приспособлены к работе вне прямой видимости.
    \end{itemize}
  \item \textbf{Полезный груз} --- за исключением БПЛА, которые служат
    испытательными образцами, большинство БПЛА выполняют определённое задание,
    которое требует наличие полезного груза на борту. Среди  полезного груза
    могут быть приборы разведки и наблюдения; оружие; коммуникационные системы;
    датчики; доставляемый груз или любой полезный груз иной категории. Некоторые
    БПЛА носят полезную нагрузку множество разных категории. БПЛА как правило
    проектируются и разрабатываются в первую очередь с учётом планируемого
    полезного груза, который будет применятся в полётных заданиях БПЛА.
  \item \textbf{Человеческий фактор} --- с каждым днём новые технологии
    позволяют минимизировать участие человека в процессе управления БПЛА. Тем не
    менее, человеческий фактор по сей день остаётся очень важным и неотменяемым
    элементом в системах БПЛА. Данный элемент состоит из оператора (пилота), и
    вспомогательной наземной команды. В зависимости от сложности системы, роли
    могут занимать несколько человек или объединить несколько ролей в одну.
\end{itemize}

\begin{figure}[!h]
  \caption{Ручной пульт радиоуправления FrSky X9D Taranis plus 2.4 GHz
  (\url{https://commons.wikimedia.org/wiki/File:FrSky_X9D_Taranis_plus_2.4_GHz_handheld_RC_transmitter.jpg}).
  Автор Enigmasoldier. Лицензия
  \href{https://creativecommons.org/licenses/by-sa/3.0/deed.en}{CC BY-SA 3.0}
  }\label{pic:rc-tx}
  \centering
  \includegraphics[width=0.5\textwidth]{rc-tx}
\end{figure}

\begin{figure}[!h]
  \caption{Внутри НСУ беспилотника RQ-7A Shadow 200
  (\url{https://commons.wikimedia.org/wiki/File:Meingcs.jpg}).
  Автор LucasBosch. Лицензия
  \href{https://creativecommons.org/licenses/by-sa/3.0/deed.en}{CC BY-SA 3.0}
  }\label{pic:gcs}
  \centering
  \includegraphics[width=0.5\textwidth]{gcs}
\end{figure}

\chapterconclusion

%% Макрос для заключения. Совместим со старым стилевиком.
\startconclusionpage

В данном разделе размещается заключение.

\printmainbibliography
\nocite{*}

%% После этой команды chapter будет генерировать приложения, нумерованные русскими буквами.
%% \startappendices из старого стилевика будет делать то же самое
\appendix

\chapter{Перечень сокращений}\label{sec:app:1}

\begin{table}[!h]
  \centering
  \begin{tabularx}{\textwidth}{L{0.5} L{1.5}}
    \hline
    ЛА & Летательный аппарат \\
    ПЛА & Пилотируемый летательный аппарат \\
    БПЛА & Беспилотный летательный аппарат \\
    НСУ & Наземная станция управления \\
    ПО & Программное обеспечение \\
    \hline
  \end{tabularx}
\end{table}

\end{document}
