\documentclass{itmo-student-thesis}

%% Опции пакета:
%% - specification - если есть, генерируется задание, иначе не генерируется
%% - annotation - если есть, генерируется аннотация, иначе не генерируется
%% - times - делает все шрифтом Times New Roman, собирается с помощью xelatex
%% - languages={...} - устанавливает перечень используемых языков. По умолчанию это {english,russian}.
%%                     Последний из языков определяет текст основного документа.

%% Делает запятую в формулах более интеллектуальной, например:
%% $1,5x$ будет читаться как полтора икса, а не один запятая пять иксов.
%% Однако если написать $1, 5x$, то все будет как прежде.
\usepackage{icomma}

%% Один из пакетов, позволяющий делать таблицы на всю ширину текста.
\usepackage{tabularx}

%% Данные пакеты необязательны к использованию в бакалаврских/магистерских
%% Они нужны для иллюстративных целей
%% Начало
\usepackage{tikz}
\usetikzlibrary{arrows}
\usepackage{filecontents}
\begin{filecontents}{bachelor-thesis.bib}
\end{filecontents}
%% Конец

%% Указываем файл с библиографией.
\addbibresource{bachelor-thesis.bib}

\begin{document}

\studygroup{K3400}
\faculty{ИКТ}
\specialty{11.03.02~Инфокоммуникационные технологии и системы свзяи}
\specialization{Программно-защищённые инфокоммуникации}
\title{Разработка программы автоматизированного построения полётного задания для
управления группой БПЛА}
\author{Де Ла Пенья Смирнов Ярослав}{Де Ла Пенья Смирнов Я.}
\supervisor{Карманов Андрей Геннадиевич}{Карманов А.Г.}{доцент, к.т.н.}{}
\programhead{Присяжнюк С.П.}{профессор, д.т.н.}
\publishyear{2020}
%% Дата выдачи задания. Можно не указывать, тогда надо будет заполнить от руки.
\startdate{20}{декабря}{2019}
%% Срок сдачи студентом работы. Можно не указывать, тогда надо будет заполнить от руки.
\finishdate{18}{мая}{2020}
%% Дата защиты. Можно не указывать, тогда надо будет заполнить от руки.
\defencedate{05}{июня}{2020}

%% \addconsultant{}{звание}

%% \secretary{}

%% Эта команда генерирует титульный лист и аннотацию.
\maketitle{Бакалавр}

%% Оглавление
\tableofcontents

%% Макрос для введения. Совместим со старым стилевиком.
\startprefacepage

В данном разделе размещается введение.

%% Начало содержательной части.
\chapter{Первая глава}

Это первая глава

%% Макрос для заключения. Совместим со старым стилевиком.
\startconclusionpage

В данном разделе размещается заключение.

\printmainbibliography

%% После этой команды chapter будет генерировать приложения, нумерованные русскими буквами.
%% \startappendices из старого стилевика будет делать то же самое
\appendix

\chapter{Перечень сокращений}\label{sec:app:1}

В данном приложении список сокращений.

\end{document}
